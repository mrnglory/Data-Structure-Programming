%%%%%%%%
% Journal Article
% LaTeX Template
% Version 1.4 (15/5/16)
%
% This template has been downloaded from:
% http://www.LaTeXTemplates.com
%
% Original author:
% Frits Wenneker (http://www.howtotex.com) with extensive modifications by
% Vel (vel@LaTeXTemplates.com)
%
% License:
% CC BY-NC-SA 3.0 (http://creativecommons.org/licenses/by-nc-sa/3.0/)
%
%%%%%%%%%%%%%%%%%%%%%%%%%%%%%%%%%%%%%%%%%

%----------------------------------------------------------------------------------------
%	PACKAGES AND OTHER DOCUMENT CONFIGURATIONS
%----------------------------------------------------------------------------------------

\documentclass[twoside,twocolumn]{article}
\usepackage{graphicx}

\usepackage{color}
\usepackage[utf8]{inputenc}
\usepackage[hangul]{kotex}

\newenvironment{itemizeReduced}{
\begin{list}{\labelitemi}{\leftmargin=1em}
\setlength{\itemsep}{1pt}
\setlength{\parskip}{0pt}
\setlength{\parsep}{0pt}}{\end{list}
}

\usepackage{listings}
\usepackage{xcolor}
\lstset{
      %backgroundcolor = \color{lightgray},
      language = C++,
      numbers=left,
      stepnumber=1,
      basicstyle=\ttfamily,
      columns=fullflexible,
      breaklines=true,
      postbreak=\mbox{\textcolor{pink}{$\hookrightarrow$}\space}
      escapeinside={\%*}{*)},          % if you want to add LaTeX within your code
      frame=tb, % draw a frame at the top and bottom of the code block
      tabsize=4, % tab space width
      showstringspaces=false, % don't mark spaces in strings
      commentstyle=\color{yellow}, % comment color
      keywordstyle=\color{blue}, % keyword color
      stringstyle=\color{green},
}

\usepackage{blindtext} % Package to generate dummy text throughout this template 

\usepackage[sc]{mathpazo} % Use the Palatino font
\usepackage[T1]{fontenc} % Use 8-bit encoding that has 256 glyphs
\linespread{1.13} % Line spacing - Palatino needs more space between lines
\usepackage{microtype} % Slightly tweak font spacing for aesthetics

\usepackage[english]{babel} % Language hyphenation and typographical rules

\usepackage[hmarginratio=1:1,top=32mm,columnsep=20pt]{geometry} % Document margins
\usepackage[hang, small,labelfont=bf,up,textfont=it,up]{caption} % Custom captions under/above floats in tables or figures
\usepackage{booktabs} % Horizontal rules in tables

\usepackage{lettrine} % The lettrine is the first enlarged letter at the beginning of the text

\usepackage{enumitem} % Customized lists
\setlist[itemize]{noitemsep} % Make itemize lists more compact

\usepackage{abstract} % Allows abstract customization
\renewcommand{\abstractnamefont}{\normalfont\bfseries} % Set the "Abstract" text to bold
\renewcommand{\abstracttextfont}{\normalfont\small\itshape} % Set the abstract itself to small italic text

\usepackage{titlesec} % Allows customization of titles
\renewcommand\thesection{\Roman{section}} % Roman numerals for the sections
\renewcommand\thesubsection{\roman{subsection}} % roman numerals for subsections
\titleformat{\section}[block]{\large\scshape\centering}{\thesection.}{1em}{} % Change the look of the section titles
\titleformat{\subsection}[block]{\large}{\thesubsection.}{1em}{} % Change the look of the section titles

\usepackage{fancyhdr} % Headers and footers
\pagestyle{fancy} % All pages have headers and footers
\fancyhead{} % Blank out the default header
\fancyfoot{} % Blank out the default footer
% Custom header text
\fancyfoot[RO,LE]{\thepage} % Custom footer text

\usepackage{titling} % Customizing the title section

\usepackage{hyperref} % For hyperlinks in the PDF

%----------------------------------------------------------------------------------------
%	TITLE SECTION
%----------------------------------------------------------------------------------------

\setlength{\droptitle}{-4\baselineskip} % Move the title up

\pretitle{\begin{center}\Huge\bfseries} % Article title formatting
\posttitle{\end{center}} % Article title closing formatting
\title{자료구조 HW5} % Article title
\author{%
\normalsize B711222 \textsc{박조은} \\ % Your name
\normalsize Hongik University \\ % Your institution
\normalsize \href{mailto:mrnglory@mail.hongik.ac.kr}{mrnglory@mail.hongik.ac.kr} % Your email address
%\and % Uncomment if 2 authors are required, duplicate these 4 lines if more
%\textsc{Jane Smith}\thanks{Corresponding author} \\[1ex] % Second author's name
%\normalsize University of Utah \\ % Second author's institution
%\normalsize \href{mailto:jane@smith.com}{jane@smith.com} % Second author's email address
}
\date{\today} % Leave empty to omit a date
\renewcommand{\maketitlehookd}{%

}

%----------------------------------------------------------------------------------------

\begin{document}

% Print the title
\maketitle

%----------------------------------------------------------------------------------------
%	ARTICLE CONTENTS
%----------------------------------------------------------------------------------------

\section{List of Source Files}
\begin{itemizeReduced}
\item hw5
\begin{itemizeReduced}
    \item hw5.cpp
    \item post.cpp
    \item post.h
    \end{itemizeReduced}
\end{itemizeReduced}
%------------------------------------------------

\section{hw5}

\subsection{hw5.cpp}
\begin{lstlisting} [basicstyle=\footnotesize]
#include <iostream>
#include "post.h"
using namespace std;

void PostFix(Expression);

int main()
{
	char line[MAXLEN];

	while (cin.getline(line, MAXLEN))
	{
		Expression e(line);	// line 버퍼를 이용하여 Expression 을 만듦

		try {
			PostFix(e);
		} catch (char const *msg) {
			cerr << "Exception: " << msg << endl;
		}
	}
}
\end{lstlisting}

\subsection{post.cpp}
\begin{lstlisting} [basicstyle=\footnotesize]
#include <iostream>
#include <stack>
#include "post.h"
using namespace std;

bool Token::operator == (char b)
{
	return len == 1 && str[0] == b;
}

bool Token::operator != (char b)
{
	return len != 1 || str[0] != b;
}

Token::Token() {}

Token::Token(char c) : len(1), type(c)
{
	str = new char[1];
	str[0] = c;	// default type = c itself
}

Token::Token(char c, char c2, int ty) : len(2), type(ty)
{
	str = new char[2];
	str[0] = c;
	str[1] = c2;
}

Token::Token(char *arr, int l, int ty = ID) : len(l), type(ty)
{
	str = new char[len + 1];

	for (int i = 0; i < len; i++)
		str[i] = arr[i];

	str[len] = '\0';

	if (type == NUM)
	{
		ival == arr[0] - '0';

		for (int i = 1; i < len; i++)
			ival = ival * 10 + arr[i] - '0';
	}

	else if (type == ID)
		ival = 0;

	else
		throw "must be ID or NUM";
}

bool Token::IsOperand()
{
	return type == ID || type == NUM;
}

ostream& operator << (ostream& os, Token t)
{
	if (t.type == UMINUS)
		os << "-u";

	else if (t.type == NUM)
		os << t.ival;

	else
		for (int i = 0; i < t.len; i++)
			os << t.str[i];

	os << " ";

	return ps;
}

bool GetID(Expression& e, Token& tok)
{
	char arr[MAXLEN];
	int idlen = 0;
	char c = e.str[e.pos];

	if (!(c >= 'a' && c <= 'z' || c >= 'A' && c <= 'Z'))
		return false;

	arr[idlen++] = c;
	e.pos++;

	while ((c = e.str[e.pos]) >= 'a' && c <= 'z'
		|| c >= 'A' && c <= 'Z'
		|| c >= '0' && c <= '9')
	{
		arr[idlen++] = c;
		e.pos++;
	}

	arr[idlen] = '\0';
	tok = Token(arr, idlen, ID);	// return an ID

	return true;
}

bool GetInt (Expression& e, Token& tok)
{
	char arr[MAXLEN];
	int len = 0;
	char c = e.str[e.pos];
	
	if (!(c > = '0' && c <= '9'))
		return false;
	
	arr[len++] = c;
	e.pos++;
	
	while ((c = e.str[e.pos]) >= '0' && c <= '9')
	{
		arr[len++] = c;
		e.pos++;
	}
	
	arr[len] = '\0';
	tok = Token(arr, len);
	
	return true;
}

void SkipBlanks(Expression& e)
{
	char c;

	while (e.pos < e.len && ((c = e.str[e.pos]) == ' ' || c == '\t'))
		e.pos++;
}

bool TwoCharOp(Expression& e, Token& tok)
{
	// 7가지 두 글자 토큰들 // <= >= == != && || -u 을 처리
	char c = e.str[e.pos];
	char c2 = e.str[e.pos + 1];
	int op;	// LE GE EQ NE AND OR UMINUS

	if (c == '<' && c2 == '=')
		op = LE;

	else if

	else
		return false;	// 맞는 두 글자 토큰이 아니면 false 를 return

	tok = Token (c, c2, op);
	e.pos += 2;

	return true;
}

bool OneCharOp(Expression& e, Token& tok)
{
	char c = e.str[e.pos];

	if (c == '-' || c == '!' || c == '*' || c == '/' || c == '%' || 
		c == '+' || c == '<' || c == '>' || c == '(' || c == ')' || c == '=')
	{
		tok = Token(c);
		e.pos++;

		return true;
	}

	return false;
}

Token NextToken(Expression& e, bool INFIX = true)
{
	static bool oprrFound = true;	// 종전에 연산자 발견되었다고 가정
	Token tok;
	SkipBlanks(e);	// skip blanks if any

	if (e.pos == e.len)	// No more token left in this expression
	{
		if (INFIX)
			oprrFound = true;

		return Token('#');
	}

	if (GetID(e, tok) || GetINT(e, tok))
	{
		if (INFIX)
			oprrFound = false;

		return tok;
	}

	if (TwoCharOp(e, tok) || OneCharOp(e, tok))
	{
		if (tok.type == '-' && INFIX && oprrFound)	// operator 후 - 발견
			tok = Token('-', 'u', UMINUS);	// unary minus (-u)로 바꾸시오

		if (INFIX)
			oprrFound = true;

		return tok;
	}

	throw "Illegal Character Found";
}

int icp(Token& t)
{
	// in-coming priority

	int ty = t.type;
	
	if (t.type == '(')
		return 0;
	
	else if (t.type == UMINUS || t.type == '!')
		return 1;
	
	else if (t.type == '*' || t.type == '/' || t.type == '%')
		return 2;
	
	else if (t.type == '+' || t.type == '-')
		return 3;
	
	else if (t.type == '<' || t.type == '>' || t.type == LE || t.type == GE)
		return 4;
	
	else if (t.type == EQ || t.type == NE)
		return 5;
	
	else if (t.type == AND)
		return 6;
	
	else if (t.type === OR)
		return 7;
	
	else if (t.type == '=')
		return 8;
	
	else if (t.type == '#')
		return 9;
}

int isp(Token& t)
{
	// in-stack priority
	
	if (t.type == '(')
		return 10;
	
	else
		return icp(t);
}

void PostFix(Expression e)
{
	// HINT: STL stack 이용하고, 교재의 마지막 for 문으로
	// while (stack.top() != '#') { cout << stack.top(); stack.pop();}
	// stack.pop();
	
	stack<Token> stack;
	stack.push('#');
	
	for (Token x = NextToken(e); x != '#'; x = NextToken(e))
	{
		if (x.IsOperand())
			cout << x;
		
		else if (x == ')')
		{
			for (; stack.top() != '('; stack.pop())
				cout << stack.top();
			stack.pop();
		}
		
		else
		{
			for (; isp(stack.top()) <= icp(x); stack.pop())
				cout << stack.top();
			stack.push(x);
		}
		
		while (stack.top() != '#')
		{
			cout << stack.top();
			stack.pop();
		}
		
		cout << endl;
	}
}
\end{lstlisting}

\subsection{post.h}
\begin{lstlisting} [basicstyle=\footnotesize]
#ifndef POSTFIX_H
#define POSTFIX_H
// token types for non one-char tokens
#define ID	257
#define NUM	258
#define EQ	259
#define NE	260
#define GE	261
#define LE	262
#define AND	263
#define OR	264
#define UMINUS	265
#define MAXLEN	80

struct Expression
{
	Expression (char* s): str(s), pos(0)
	{for (len = 0; str[len] != '\0'; len++);}

	char * str;
	int pos;
	int len;
};

struct Token
{
	bool operator == (char);
	bool operator != (char);

	Token();

	Token(char);		// 1-char token: type equals the token itself
	Token(char, char, int);	// 2-char token (e.g. <=) & its type (e.g. LE)
	Token(char*, int, int);	// operand with its length & type (defaulted to ID)
	bool IsOperand();	// true if the token type is ID or NUM
	int type;		// ascii code for 1-char op; predefined for other tokens
	char *str;		//token value
	int len;		// length of str
	int ival;		// used to store an integer for type NUM; init to 0 for ID
};

using namespace std;
ostream& operator << (ostream&, Token);
osteram& operator << (ostream&, Expression);
Token NextToken(Expression&, bool);	// 2nd arg = true for infix expression
void PostFix(Expression e);

#endif
\end{lstlisting}
\begin{itemizeReduced}
    \item[/*] 
    \item[*] 갓직히 왜 컴파일이 안되는지 납득이 안됨.
\end{itemizeReduced}
*/

%------------------------------------------------

\newpage


%----------------------------------------------------------------------------------------
%	REFERENCE LIST
%----------------------------------------------------------------------------------------


%----------------------------------------------------------------------------------------

\end{document}
