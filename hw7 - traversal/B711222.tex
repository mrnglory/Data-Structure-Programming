%%%%%%%%
% Journal Article
% LaTeX Template
% Version 1.4 (15/5/16)
%
% This template has been downloaded from:
% http://www.LaTeXTemplates.com
%
% Original author:
% Frits Wenneker (http://www.howtotex.com) with extensive modifications by
% Vel (vel@LaTeXTemplates.com)
%
% License:
% CC BY-NC-SA 3.0 (http://creativecommons.org/licenses/by-nc-sa/3.0/)
%
%%%%%%%%%%%%%%%%%%%%%%%%%%%%%%%%%%%%%%%%%

%----------------------------------------------------------------------------------------
%	PACKAGES AND OTHER DOCUMENT CONFIGURATIONS
%----------------------------------------------------------------------------------------

\documentclass[twoside,twocolumn]{article}
\usepackage{graphicx}

\usepackage{color}
\usepackage[utf8]{inputenc}
\usepackage[hangul]{kotex}

\newenvironment{itemizeReduced}{
\begin{list}{\labelitemi}{\leftmargin=1em}
\setlength{\itemsep}{1pt}
\setlength{\parskip}{0pt}
\setlength{\parsep}{0pt}}{\end{list}
}

\usepackage{listings}
\usepackage{xcolor}
\lstset{
      %backgroundcolor = \color{lightgray},
      language = C++,
      numbers=left,
      stepnumber=1,
      basicstyle=\ttfamily,
      columns=fullflexible,
      breaklines=true,
      postbreak=\mbox{\textcolor{pink}{$\hookrightarrow$}\space}
      escapeinside={\%*}{*)},          % if you want to add LaTeX within your code
      frame=tb, % draw a frame at the top and bottom of the code block
      tabsize=4, % tab space width
      showstringspaces=false, % don't mark spaces in strings
      commentstyle=\color{yellow}, % comment color
      keywordstyle=\color{blue}, % keyword color
      stringstyle=\color{green},
}

\usepackage{blindtext} % Package to generate dummy text throughout this template 

\usepackage[sc]{mathpazo} % Use the Palatino font
\usepackage[T1]{fontenc} % Use 8-bit encoding that has 256 glyphs
\linespread{1.13} % Line spacing - Palatino needs more space between lines
\usepackage{microtype} % Slightly tweak font spacing for aesthetics

\usepackage[english]{babel} % Language hyphenation and typographical rules

\usepackage[hmarginratio=1:1,top=32mm,columnsep=20pt]{geometry} % Document margins
\usepackage[hang, small,labelfont=bf,up,textfont=it,up]{caption} % Custom captions under/above floats in tables or figures
\usepackage{booktabs} % Horizontal rules in tables

\usepackage{lettrine} % The lettrine is the first enlarged letter at the beginning of the text

\usepackage{enumitem} % Customized lists
\setlist[itemize]{noitemsep} % Make itemize lists more compact

\usepackage{abstract} % Allows abstract customization
\renewcommand{\abstractnamefont}{\normalfont\bfseries} % Set the "Abstract" text to bold
\renewcommand{\abstracttextfont}{\normalfont\small\itshape} % Set the abstract itself to small italic text

\usepackage{titlesec} % Allows customization of titles
\renewcommand\thesection{\Roman{section}} % Roman numerals for the sections
\renewcommand\thesubsection{\roman{subsection}} % roman numerals for subsections
\titleformat{\section}[block]{\large\scshape\centering}{\thesection.}{1em}{} % Change the look of the section titles
\titleformat{\subsection}[block]{\large}{\thesubsection.}{1em}{} % Change the look of the section titles

\usepackage{fancyhdr} % Headers and footers
\pagestyle{fancy} % All pages have headers and footers
\fancyhead{} % Blank out the default header
\fancyfoot{} % Blank out the default footer
% Custom header text
\fancyfoot[RO,LE]{\thepage} % Custom footer text

\usepackage{titling} % Customizing the title section

\usepackage{hyperref} % For hyperlinks in the PDF

%----------------------------------------------------------------------------------------
%	TITLE SECTION
%----------------------------------------------------------------------------------------

\setlength{\droptitle}{-4\baselineskip} % Move the title up

\pretitle{\begin{center}\Huge\bfseries} % Article title formatting
\posttitle{\end{center}} % Article title closing formatting
\title{자료구조 HW7} % Article title
\author{%
\normalsize B711222 \textsc{박조은} \\ % Your name
\normalsize Hongik University \\ % Your institution
\normalsize \href{mailto:mrnglory@mail.hongik.ac.kr}{mrnglory@mail.hongik.ac.kr} % Your email address
%\and % Uncomment if 2 authors are required, duplicate these 4 lines if more
%\textsc{Jane Smith}\thanks{Corresponding author} \\[1ex] % Second author's name
%\normalsize University of Utah \\ % Second author's institution
%\normalsize \href{mailto:jane@smith.com}{jane@smith.com} % Second author's email address
}
\date{\today} % Leave empty to omit a date
\renewcommand{\maketitlehookd}{%

}

%----------------------------------------------------------------------------------------

\begin{document}

% Print the title
\maketitle

%----------------------------------------------------------------------------------------
%	ARTICLE CONTENTS
%----------------------------------------------------------------------------------------

\section{List of Source Files}
\begin{itemizeReduced}
\item hw7
\begin{itemizeReduced}
    \item hw7.cpp
    \item bt.h
    \end{itemizeReduced}
\end{itemizeReduced}
%------------------------------------------------

\section{hw7}

\subsection{hw7.cpp}
\begin{lstlisting} [basicstyle=\footnotesize]
#include "bt.h"
#include <iostream>
using namespace std;

int main()
{
        Tree<double> tree;
        double dval;

        cout << "Enter doubles:\n";

        while(cin >> dval)
                tree.Insert(dval);

        cout << endl << "Preorder traversal:    ";
                tree.Preorder();

        cout << endl << "Inorder traversal:     ";
                tree.Inorder();

        cout << endl << "Postorder traversal:   ";
                tree.Postorder();

        cout << endl << "Levelorder traversal:  ";
                tree.Levelorder();

        cout << endl;
}
\end{lstlisting}

\subsection{bt.h}
\begin{lstlisting} [basicstyle=\footnotesize]
#ifndef TREE_H
#define TREE_H
#include <iostream>
#include <queue>
using namespace std;

template <class T>

struct Node
{
	Node(T d, Node<T> *left = 0, Node<T> *right = 0)
		: data(d), leftChild(left), rightChild(right) {}

	Node<T> *leftChild;
	T data;
	Node<T> *rightChild;
	Node<T> *pre;
};

template <class T>

class Tree
{
public:
	Tree() {root = 0; } // empty tree

	void Insert(T &value) 
	{
		Insert(root, value); 
	}

	void Preorder()
	{
		Preorder(root);
	}

	void Inorder()
	{
		Inorder(root); 
	}

	void Postorder()
	{
		Postorder(root);
	}

	void Levelorder();

private:
	void Visit(Node<T> *);
	void Insert(Node<T> *&, T &);
	void Preorder(Node<T> *);
	void Inorder(Node<T> *);
	void Postorder(Node<T> *);

	Node<T> *root;
};

template <class T>
void Tree<T>::Visit(Node<T> *ptr)
{
	cout << ptr -> data << " ";
}

template <class T>
void Tree<T>::Insert(Node<T>* &ptr, T &value)
{
	// Insert 의 helper 함수
	if (ptr == 0)
		ptr = new Node<T>(value);

	else if (value < ptr -> data)
		Insert(ptr -> leftChild, value);

	else if (value > ptr -> data)
		Insert(ptr -> rightChild, value);

	else
		cout << endl << "Duplicate value" << value << "ignored\n";
}

// Preorder, Inorder, Postorder, Levelorder 함수를 구현하시오.
// Levelorder(교재 p266 참조하되 STL 큐를 이용) 를 구현하시오.

template <class T>
void Tree<T>::Preorder(Node<T> *currentNode)
{
	// Workhorse traverses the subtree rooted at currentNode.
	// The workhorse is declared
	// as a private member function of Tree.

	if(currentNode)
	{
		Visit(currentNode);
		Preorder(currentNode -> leftChild);
		Preorder(currentNode -> rightChild);
	}
}

template <class T>
void Tree<T>::Postorder(Node<T> *currentNode)
{
	// Workhorse traverses the subtree rooted at currentNode.
	// The workhorse is declared
	// as a private member function of Tree.

	if(currentNode)
	{
		Postorder(currentNode -> leftChild);
		Postorder(currentNode -> rightChild);
		Visit(currentNode);
	}
}

template <class T>
void Tree<T>::Inorder(Node<T> *root)
{
	// Workhorse traverses the subtree rooted at currentNode.
	// The workhorse is declared
	// as a priate member function of Tree.

	Node<T> * currentNode, * pre;
	currentNode = root;

	if(root == NULL)
		return;

	while(currentNode != NULL)
	{
		if(currentNode -> leftChild == NULL)
		{
			Visit(currentNode);
			currentNode = currentNode -> rightChild;
		}

		else
		{
			pre = currentNode -> leftChild;

			while(pre -> rightChild != NULL && pre -> rightChild != currentNode)
				pre = pre -> rightChild;

			if(pre -> rightChild == NULL)
			{
				pre -> rightChild = currentNode;
				currentNode = currentNode -> leftChild;
			}

			else
			{
				pre -> rightChild = NULL;
				Visit(currentNode);
				currentNode = currentNode -> rightChild;
			}
		}
	}
}

template <class T>
void Tree<T>::Levelorder()
{
	queue<Node<T>*> q;
	Node<T> * currentNode = root;

	while(currentNode)
	{
		Visit(currentNode);

		if(currentNode -> leftChild)
			q.push(currentNode -> leftChild);

		if(currentNode -> rightChild)
			q.push(currentNode -> rightChild);

		if(q.empty())
			return;

		currentNode = q.front(); // 큐에서 꺼내자.
		q.pop();
	}
}

#endif
\end{lstlisting}
\begin{itemizeReduced}
    \item[/*] 
    \item[*] 
    \item[*] line 9-18: Node 정의
    \item[*] line 17: Non recursion inorder traversal of Threaded binary tree 를 위해 추가한 포인터, predecessor Node
    \item[*] line 22-57: Tree 정의
    \item[*] line 60-63: Visit 시 출력 패턴 정의
    \item[*] line 66-80: 입력 값을 left 혹은 right child 로서 Insert 하는 경우를 커버하는 함수 정의
    \item[*] line 86-98: root -> left child -> right child visiting Preorder function definition
    \item[*] line 108-120: left child -> right child -> root visiting Postorder function definition
    \item[*] line 122-157: Inorder traversal of Threaded binary tree (a.k.a. Morris inorder traversal using threading)
    \item[*] line 130-134, 138: header node = most left node 의 leftChild 및 most right node 의 rightChild 가 가리키게 함.
    \item[*] line 159-181: queue 사용하여 root -> left child -> right child 방문하되, 각 레벨별로 most left  node 에서 most right node 모두를 방문함.
\end{itemizeReduced}
*/
\subsection{Results}
\subsubsection{makefile}
\begin{lstlisting} [basicstyle=\footnotesize]
hw7:    hw7.o
        g++ -o hw7 hw7.o
hw7.o:  bt.h
\end{lstlisting}

\subsubsection{compile}
\begin{lstlisting} [basicstyle=\footnotesize]
[B711222@localhost hw7d]$ hw7
Enter doubles:
35.3 15.7 81.5 4.5 66.7 91.2 2.3 5.2 88.2 94.5

Preorder traversal:     35.3 15.7 4.5 2.3 5.2 81.5 66.7 91.2 88.2 94.5
Inorder traversal:      2.3 4.5 5.2 15.7 35.3 66.7 81.5 88.2 91.2 94.5
Postorder traversal:    2.3 5.2 4.5 15.7 66.7 88.2 94.5 91.2 81.5 35.3
Levelorder traversal:   35.3 15.7 81.5 4.5 66.7 91.2 2.3 5.2 88.2 94.5
\end{lstlisting}

%------------------------------------------------

\newpage


%----------------------------------------------------------------------------------------
%	REFERENCE LIST
%----------------------------------------------------------------------------------------


%----------------------------------------------------------------------------------------

\end{document}
