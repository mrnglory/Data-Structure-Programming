%%%%%%%%
% Journal Article
% LaTeX Template
% Version 1.4 (15/5/16)
%
% This template has been downloaded from:
% http://www.LaTeXTemplates.com
%
% Original author:
% Frits Wenneker (http://www.howtotex.com) with extensive modifications by
% Vel (vel@LaTeXTemplates.com)
%
% License:
% CC BY-NC-SA 3.0 (http://creativecommons.org/licenses/by-nc-sa/3.0/)
%
%%%%%%%%%%%%%%%%%%%%%%%%%%%%%%%%%%%%%%%%%

%----------------------------------------------------------------------------------------
%	PACKAGES AND OTHER DOCUMENT CONFIGURATIONS
%----------------------------------------------------------------------------------------

\documentclass[twoside,twocolumn]{article}

\usepackage{color}
\usepackage[utf8]{inputenc}
\usepackage[hangul]{kotex}

\newenvironment{itemizeReduced}{
\begin{list}{\labelitemi}{\leftmargin=1em}
\setlength{\itemsep}{1pt}
\setlength{\parskip}{0pt}
\setlength{\parsep}{0pt}}{\end{list}
}

\usepackage{listings}
\usepackage{xcolor}
\lstset{
      %backgroundcolor = \color{lightgray},
      language = C++,
      numbers=left,
      stepnumber=1,
      basicstyle=\ttfamily,
      columns=fullflexible,
      breaklines=true,
      postbreak=\mbox{\textcolor{pink}{$\hookrightarrow$}\space}
      escapeinside={\%*}{*)},          % if you want to add LaTeX within your code
      frame=tb, % draw a frame at the top and bottom of the code block
      tabsize=4, % tab space width
      showstringspaces=false, % don't mark spaces in strings
      commentstyle=\color{yellow}, % comment color
      keywordstyle=\color{blue}, % keyword color
      stringstyle=\color{green},
}

\usepackage{blindtext} % Package to generate dummy text throughout this template 

\usepackage[sc]{mathpazo} % Use the Palatino font
\usepackage[T1]{fontenc} % Use 8-bit encoding that has 256 glyphs
\linespread{1.13} % Line spacing - Palatino needs more space between lines
\usepackage{microtype} % Slightly tweak font spacing for aesthetics

\usepackage[english]{babel} % Language hyphenation and typographical rules

\usepackage[hmarginratio=1:1,top=32mm,columnsep=20pt]{geometry} % Document margins
\usepackage[hang, small,labelfont=bf,up,textfont=it,up]{caption} % Custom captions under/above floats in tables or figures
\usepackage{booktabs} % Horizontal rules in tables

\usepackage{lettrine} % The lettrine is the first enlarged letter at the beginning of the text

\usepackage{enumitem} % Customized lists
\setlist[itemize]{noitemsep} % Make itemize lists more compact

\usepackage{abstract} % Allows abstract customization
\renewcommand{\abstractnamefont}{\normalfont\bfseries} % Set the "Abstract" text to bold
\renewcommand{\abstracttextfont}{\normalfont\small\itshape} % Set the abstract itself to small italic text

\usepackage{titlesec} % Allows customization of titles
\renewcommand\thesection{\Roman{section}} % Roman numerals for the sections
\renewcommand\thesubsection{\roman{subsection}} % roman numerals for subsections
\titleformat{\section}[block]{\large\scshape\centering}{\thesection.}{1em}{} % Change the look of the section titles
\titleformat{\subsection}[block]{\large}{\thesubsection.}{1em}{} % Change the look of the section titles

\usepackage{fancyhdr} % Headers and footers
\pagestyle{fancy} % All pages have headers and footers
\fancyhead{} % Blank out the default header
\fancyfoot{} % Blank out the default footer
% Custom header text
\fancyfoot[RO,LE]{\thepage} % Custom footer text

\usepackage{titling} % Customizing the title section

\usepackage{hyperref} % For hyperlinks in the PDF

%----------------------------------------------------------------------------------------
%	TITLE SECTION
%----------------------------------------------------------------------------------------

\setlength{\droptitle}{-4\baselineskip} % Move the title up

\pretitle{\begin{center}\Huge\bfseries} % Article title formatting
\posttitle{\end{center}} % Article title closing formatting
\title{자료구조 HW4} % Article title
\author{%
\normalsize B711222 \textsc{박조은} \\ % Your name
\normalsize Hongik University \\ % Your institution
\normalsize \href{mailto:mrnglory@mail.hongik.ac.kr}{mrnglory@mail.hongik.ac.kr} % Your email address
%\and % Uncomment if 2 authors are required, duplicate these 4 lines if more
%\textsc{Jane Smith}\thanks{Corresponding author} \\[1ex] % Second author's name
%\normalsize University of Utah \\ % Second author's institution
%\normalsize \href{mailto:jane@smith.com}{jane@smith.com} % Second author's email address
}
\date{\today} % Leave empty to omit a date
\renewcommand{\maketitlehookd}{%

}

%----------------------------------------------------------------------------------------

\begin{document}

% Print the title
\maketitle

%----------------------------------------------------------------------------------------
%	ARTICLE CONTENTS
%----------------------------------------------------------------------------------------

\section{List of Source Files}
\begin{itemizeReduced}
\item hw4
\begin{itemizeReduced}
    \item maze.cpp
    \item hw4.cpp
    \item makefile
    \item maze.in
    \item maze.in2
    \end{itemizeReduced}
\end{itemizeReduced}
%------------------------------------------------

\section{hw4}

\subsection{maze.cpp}
\begin{lstlisting} [basicstyle=\footnotesize]
#include <iostream>
#include <stack>
using namespace std;

const int MAXSIZE = 100; // up to 100 by 100 maze allowed
bool maze[MAXSIZE + 2][MAXSIZE + 2];
bool mark[MAXSIZE + 1][MAXSIZE + 1] = {0};

enum directions {N, NE, E, SE, S, SW, W, NW};

struct offsets
{
        int a, b;
} move[8] = {-1,0, -1,1, 0,1, 1,1, 1,0, 1,-1, 0,-1, -1,-1};

struct Items
{
        Items(int xx = 0, int yy = 0, int dd = 0): x(xx), y(yy), dir(dd) {}
                int x, y, dir;
};

template <class T>
ostream& operator<< (ostream& os, stack<T>& s)
{
        // 스택의 내용을 역순으로 출력
        // 구현방법 = 내용을 하나씩 꺼내 다른 임시 스택에 넣어 저장한 후,
        // 최종적으로 그 임시 스택에서 하나씩 꺼내 출력하면 됨

        //os << "top = " << s.top() << endl;
        //for (int i = 0; i <= s.top(); i++)
        //      os << i << ":" << s.stack[i] << endl;

        stack<T> s2;

        while (!s.empty())
        {
                s2.push(s.top());
                s.pop();
        }

        while (!s2.empty())
        {
                os << " -> " << s2.top();
                s2.pop();
        }

        return os;
}

ostream& operator<< (ostream& os, Items& item)
{
        // 5개의 Items가 출력 될 때마다 줄 바꾸기 위해

        static int count = 0;

        os << "(" << item.x << ", " << item.y << ")";
        count++;

        if ((count % 5) == 0)
                os << endl;

        return os;
}

void Path (const int m, const int p)
{
        // 구현은 책과 동일하다. 단, 최종적인 경로의 출력은 다음과 같이 한다.

        mark[1][1] = 1; // start at (1, 1)
        stack<Items> stack;
        Items temp(1, 1, E);
        stack.push(temp);

        while (!stack.empty())
        {
                temp = stack.top();
                stack.pop(); // unstack

                int i = temp.x;
                int j = temp.y;
                int d = temp.dir;

                while (d < 8) // move forward
                { // (i, j)에서 (g, h)로 이동한다고 하자.
                        int g = i + move[d].a;
                        int h = j + move[d].b;

                        if ((g == m) && (h == p))
                        {
                                int node = 0;

                                cout << stack;

                                temp.x = i;
                                temp.y = j;
                                cout << " -> " << temp;

                                temp.x = m;
                                temp.y = p;
                                cout << " -> " << temp << endl;

                                for (int i = 1; i < m + 1; i++)
                                        for (int j = 1; j < p + 1; j++)
                                                if (mark[i][j])
                                                        node++;
                                cout << "#nodes visited = " << node << " out of              " << m * p << endl;

                                return;
                        }

                        if ((!maze[g][h]) && (!mark[g][h])) // new position
                        {
                                mark[g][h] = 1; // 방문한 적이 있다고 표시

                                temp.x = i;
                                temp.y = j;
                                temp.dir = d + 1; // 현 위치 및 실패 시 다음에              시도할 방향 저장

                                stack.push(temp); // stack it
                                i = g;
                                j = h;
                                d = N; // N방향부터 (시계방향으로) 시도하자
                        }
                        else
                                d++; // try next direction
                } // end of while (d < 8)
        } // end of while (!stack.empty())
        cout << "No path in maze." << endl;
}

void getdata(istream& is, int& m, int& p)
{ // 자료파일을 읽어들여 maze 에 저장한다.
        is >> m >> p;

        for (int i = 0; i < m + 2; i++)
        { // 왼쪽 벽과 오른쪽 벽 작성
                maze[i][0] = 1;
                maze[i][p + 1] = 1;
        }

        for (int j = 1; j <= p; j++)
        { // 윗쪽 벽과 아랫쪽 벽 작성
                maze[0][j] = 1;
                maze[m + 1][j] = 1;
        }

        for (int i = 1; i <= m; i++) // 자료 읽어들이기
                for (int j = 1; j <= p; j++)
                        is >> maze[i][j];
}
\end{lstlisting}
\begin{itemizeReduced}
    \item[/*] 
    \item[*] line 6: 경계선에 있는 경우, 즉 i == 1, i == m, j == 1, j == p 인 경우, 벽쪽으로는 진행하지 못하므로 가능한 방향은 8방향보다 작다. 따라서 경계조건을 매번 검사하지 않기 위해 미로의 주위를 1로 둘러싼다. 배열의 크기를 maze[m + 2][p + 2]로 선언한 이유가 이것이다.
    \item[*] line 11 - 14: 미로 이동 시, 현재의 위치와 직전 이동 방향을 저장한 후, 한 방향을 선택한다. 북쪽부터 시작하여 시계방향으로 8가지의 방향을 배열 move에 정의해준다.
    \item[*] line 50: Stack 과 Items 에 대해 연산자 다중화를 진행한다. friend 로 선언되어 Stack 의 전용 데이터 멤버 접근이 가능해진다.
    \item[*] line 83: 갈 수 있는 모든 방향을 고려한다.
    \item[*] line 102 - 105: 방문한 node 의 횟수를 카운팅 하기 위해 중첩반복문으로 표현하였다. i 와 j 가 각각 1 부터 m 까지, 1 부터 p 까지만 반복되는 이유는 앞서 말했듯이 미로의 경계를 1로 둘러쌌기 때문에 해당 부분을 배제한 것이다.
    \item[*] line 90: int 형 변수 node 를 0으로 초기화 해주었다.
    \item[*] line 113: 이미 갔던 길을 다시 방문하지 않기 위함.
    \item[*] 
\end{itemizeReduced}
*/

\subsection{hw4.cpp}
\begin{lstlisting} [basicstyle=\footnotesize]
#include <iostream>
#include <fstream>
#include <cstdlib>
using namespace std;

void getdata(istream&, int&, int&);
void Path(int, int);

int main(int argc, char* argv[])
{
        int m, p; // m by p maze

        if (argc == 1)
                cerr << "Usage: " << argv[0] << "maze_data_file\n" << endl;
        else
        {
                ifstream is(argv[1]);
                if (!is)
                {
                        cerr << argv[1] << " does not exist\n";
                        exit(1);
                }

                cout << "For maze datafile (" << argv[1] << ")\n";

                getdata(is, m, p);
                is.close();
                Path(m, p);
        }
}
\end{lstlisting}
\begin{itemizeReduced}
    \item[/*] 
    \item[*] line 6: 미로의 정보를 읽어들인다. 해당 코드는 maze.cpp 의 line 131 - 150 에 구현되어있다.
    \item[*] line 7: 경로 설정을 위한 코드.
\end{itemizeReduced}
*/

%------------------------------------------------

\newpage

\subsection{Results}
\subsubsection{makefile}
\begin{lstlisting} [basicstyle=\footnotesize]
hw4:    hw4.o   maze.o
        g++ -o hw4 hw4.o maze.o
\end{lstlisting}

\subsubsection{maze.in}
\begin{lstlisting} [basicstyle=\footnotesize]
12  15
0 1 0 0 0 1 1 0 0 0 1 1 1 1 1
1 0 0 0 1 1 0 1 1 1 0 0 1 1 1
0 1 1 0 0 0 0 1 1 1 1 0 0 1 1
1 1 0 1 1 1 1 0 1 1 0 1 1 0 0
1 1 0 1 0 0 1 0 1 1 1 1 1 1 1
0 0 1 1 0 1 1 1 0 1 0 0 1 0 1
0 0 1 1 0 1 1 1 0 1 0 0 1 0 1
0 1 1 1 1 0 0 1 1 1 1 1 1 1 1
0 0 1 1 0 1 1 0 1 1 1 1 1 0 1
1 1 0 0 0 1 1 0 1 1 0 0 0 0 0
0 0 1 1 1 1 1 0 0 0 1 1 1 1 0
0 1 0 0 1 1 1 1 1 0 1 1 1 1 0
\end{lstlisting}

\subsubsection{maze.in2}
\begin{lstlisting} [basicstyle=\footnotesize]
9 9
0 0 0 0 0 0 0 0 1
1 1 1 1 1 1 1 1 0
1 0 0 0 0 0 0 0 1
0 1 1 1 1 1 1 1 1
1 0 0 0 0 0 0 0 1
1 1 1 1 1 1 1 1 0
1 0 0 0 0 0 0 0 1
0 1 1 1 1 1 1 1 1
1 0 0 0 0 0 0 0 0
\end{lstlisting}

\subsubsection{compile}
\begin{lstlisting} [basicstyle=\footnotesize]
[B711222@localhost hw4d]$ hw4 maze.in
For maze datafile (maze.in)
 -> (1, 1) -> (2, 2) -> (1, 3) -> (1, 4) -> (1, 5)
 -> (2, 4) -> (3, 5) -> (3, 4) -> (4, 3) -> (5, 3)
 -> (6, 2) -> (7, 2) -> (8, 1) -> (9, 2) -> (10, 3)
 -> (10, 4) -> (9, 5) -> (8, 6) -> (8, 7) -> (9, 8)
 -> (10, 8) -> (11, 9) -> (11, 10) -> (10, 11) -> (10, 12)
 -> (10, 13) -> (9, 14) -> (10, 15) -> (11, 15) -> (12, 15)

#nodes visited = 48 out of 180
\end{lstlisting}

\begin{lstlisting} [basicstyle=\footnotesize]
[B711222@localhost hw4d]$ hw4 maze.in2
For maze datafile (maze.in2)
 -> (1, 1) -> (1, 2) -> (1, 3) -> (1, 4) -> (1, 5)
 -> (1, 6) -> (1, 7) -> (1, 8) -> (2, 9) -> (3, 8)
 -> (3, 7) -> (3, 6) -> (3, 5) -> (3, 4) -> (3, 3)
 -> (3, 2) -> (4, 1) -> (5, 2) -> (5, 3) -> (5, 4)
 -> (5, 5) -> (5, 6) -> (5, 7) -> (5, 8) -> (6, 9)
 -> (7, 8) -> (7, 7) -> (7, 6) -> (7, 5) -> (7, 4)
 -> (7, 3) -> (7, 2) -> (8, 1) -> (9, 2) -> (9, 3)
 -> (9, 4) -> (9, 5) -> (9, 6) -> (9, 7) -> (9, 8)
 -> (9, 9)
#nodes visited = 40 out of 81
\end{lstlisting}
\begin{itemizeReduced}
    \item[/*] 어려웠던 점
    \item[*] 처음에는 미로의 입구와 출구는 잘 출력이 되는 반면에 이동 경로가 조금씩 다르게 나왔었다.
    \item[*] 아무리 생각해도 모르겠어서 그냥 포기하려고 했는데, maze.cpp 의 move[8] 배열에 진행방향 저장을 할 때, enum directions 에 저장되어있는 방향들과 맞춰서 동일하게 작성을 해야 했는데, 원소들을 그냥 아무렇게나 적었기 때문에 이동경로가 조금씩 다르게 나왔었다는 걸 깨달았다.
\end{itemizeReduced}
*/
%----------------------------------------------------------------------------------------
%	REFERENCE LIST
%----------------------------------------------------------------------------------------


%----------------------------------------------------------------------------------------

\end{document}
