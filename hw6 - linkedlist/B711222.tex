%%%%%%%%
% Journal Article
% LaTeX Template
% Version 1.4 (15/5/16)
%
% This template has been downloaded from:
% http://www.LaTeXTemplates.com
%
% Original author:
% Frits Wenneker (http://www.howtotex.com) with extensive modifications by
% Vel (vel@LaTeXTemplates.com)
%
% License:
% CC BY-NC-SA 3.0 (http://creativecommons.org/licenses/by-nc-sa/3.0/)
%
%%%%%%%%%%%%%%%%%%%%%%%%%%%%%%%%%%%%%%%%%

%----------------------------------------------------------------------------------------
%	PACKAGES AND OTHER DOCUMENT CONFIGURATIONS
%----------------------------------------------------------------------------------------

\documentclass[twoside,twocolumn]{article}
\usepackage{graphicx}

\usepackage{color}
\usepackage[utf8]{inputenc}
\usepackage[hangul]{kotex}

\newenvironment{itemizeReduced}{
\begin{list}{\labelitemi}{\leftmargin=1em}
\setlength{\itemsep}{1pt}
\setlength{\parskip}{0pt}
\setlength{\parsep}{0pt}}{\end{list}
}

\usepackage{listings}
\usepackage{xcolor}
\lstset{
      %backgroundcolor = \color{lightgray},
      language = C++,
      numbers=left,
      stepnumber=1,
      basicstyle=\ttfamily,
      columns=fullflexible,
      breaklines=true,
      postbreak=\mbox{\textcolor{pink}{$\hookrightarrow$}\space}
      escapeinside={\%*}{*)},          % if you want to add LaTeX within your code
      frame=tb, % draw a frame at the top and bottom of the code block
      tabsize=4, % tab space width
      showstringspaces=false, % don't mark spaces in strings
      commentstyle=\color{yellow}, % comment color
      keywordstyle=\color{blue}, % keyword color
      stringstyle=\color{green},
}

\usepackage{blindtext} % Package to generate dummy text throughout this template 

\usepackage[sc]{mathpazo} % Use the Palatino font
\usepackage[T1]{fontenc} % Use 8-bit encoding that has 256 glyphs
\linespread{1.13} % Line spacing - Palatino needs more space between lines
\usepackage{microtype} % Slightly tweak font spacing for aesthetics

\usepackage[english]{babel} % Language hyphenation and typographical rules

\usepackage[hmarginratio=1:1,top=32mm,columnsep=20pt]{geometry} % Document margins
\usepackage[hang, small,labelfont=bf,up,textfont=it,up]{caption} % Custom captions under/above floats in tables or figures
\usepackage{booktabs} % Horizontal rules in tables

\usepackage{lettrine} % The lettrine is the first enlarged letter at the beginning of the text

\usepackage{enumitem} % Customized lists
\setlist[itemize]{noitemsep} % Make itemize lists more compact

\usepackage{abstract} % Allows abstract customization
\renewcommand{\abstractnamefont}{\normalfont\bfseries} % Set the "Abstract" text to bold
\renewcommand{\abstracttextfont}{\normalfont\small\itshape} % Set the abstract itself to small italic text

\usepackage{titlesec} % Allows customization of titles
\renewcommand\thesection{\Roman{section}} % Roman numerals for the sections
\renewcommand\thesubsection{\roman{subsection}} % roman numerals for subsections
\titleformat{\section}[block]{\large\scshape\centering}{\thesection.}{1em}{} % Change the look of the section titles
\titleformat{\subsection}[block]{\large}{\thesubsection.}{1em}{} % Change the look of the section titles

\usepackage{fancyhdr} % Headers and footers
\pagestyle{fancy} % All pages have headers and footers
\fancyhead{} % Blank out the default header
\fancyfoot{} % Blank out the default footer
% Custom header text
\fancyfoot[RO,LE]{\thepage} % Custom footer text

\usepackage{titling} % Customizing the title section

\usepackage{hyperref} % For hyperlinks in the PDF

%----------------------------------------------------------------------------------------
%	TITLE SECTION
%----------------------------------------------------------------------------------------

\setlength{\droptitle}{-4\baselineskip} % Move the title up

\pretitle{\begin{center}\Huge\bfseries} % Article title formatting
\posttitle{\end{center}} % Article title closing formatting
\title{자료구조 HW6} % Article title
\author{%
\normalsize B711222 \textsc{박조은} \\ % Your name
\normalsize Hongik University \\ % Your institution
\normalsize \href{mailto:mrnglory@mail.hongik.ac.kr}{mrnglory@mail.hongik.ac.kr} % Your email address
%\and % Uncomment if 2 authors are required, duplicate these 4 lines if more
%\textsc{Jane Smith}\thanks{Corresponding author} \\[1ex] % Second author's name
%\normalsize University of Utah \\ % Second author's institution
%\normalsize \href{mailto:jane@smith.com}{jane@smith.com} % Second author's email address
}
\date{\today} % Leave empty to omit a date
\renewcommand{\maketitlehookd}{%

}

%----------------------------------------------------------------------------------------

\begin{document}

% Print the title
\maketitle

%----------------------------------------------------------------------------------------
%	ARTICLE CONTENTS
%----------------------------------------------------------------------------------------

\section{List of Source Files}
\begin{itemizeReduced}
\item hw6
\begin{itemizeReduced}
    \item hw6.cpp
    \item list.h
    \item list.cpp
    \item makefile
    \item input.in
    \end{itemizeReduced}
\end{itemizeReduced}
%------------------------------------------------

\section{hw6}

\subsection{hw6.cpp}
\begin{lstlisting} [basicstyle=\footnotesize]
#include <iostream>
#include "list.h"
using namespace std;

int main()
{
	IntList il;
	int input;

	cout << "============ Input =============" << endl;

		while (1)
		{
			// -1 을 받을 때 까지 반복
			cin >> input;

			if (input == -1)
				break;

			il.Insert(input);

			cout << il;
		}

	cout << "============= Delete ============" << endl;

		input = 0;

		while (1)
		{
			// -1 을 받을 때 까지 반복
			cin >> input;

			if (input == -1)
				break;

			il.Delete(input);

			cout << il;
	}

	// Push_Front 와 Push_Back 은 2회씩.
	cout << "========== Push_Front ===========" << endl;

		cin >> input;
		il.Push_Front(input);
		cout << il;

		cin >> input;
		il.Push_Front(input);
		cout << il;

	cout << "=========== Push_Back ===========" << endl;

		cin >> input;
		il.Push_Back(input);
		cout << il;

		cin >> input;
		il.Push_Back(input);
		cout << il;

	return 0;
}
\end{lstlisting}

\subsection{list.h}
\begin{lstlisting} [basicstyle=\footnotesize]
#include <iostream>
using namespace std;

struct Node
{
	Node (int d = 0, Node* ptr = NULL) : data(d), link(ptr) {}

	int data;
	Node* link;
};

struct IntList
{
	IntList()
	{
		last = first = NULL;
	}

	void Push_Back(int);	// 리스트 뒤에 삽입 중복 허용
	void Push_Front(int);	// 리스트 앞에 삽입 중복 허용
	void Insert(int);	// 정렬되어있다는 가정 하에 제 위치에 삽입
	void Delete(int);	// 리스트의 원소 삭제
	Node *first;		// 첫 노드를 가리킴
	Node *last;		// 마지막 노드를 가리킴
};

ostream& operator << (ostream&, IntList&);
\end{lstlisting}

\subsection{list.cpp}
\begin{lstlisting} [basicstyle=\footnotesize]
#include <iostream>
#include "list.h"

ostream& operator << (ostream& os, IntList& il)
{
	Node *ptr = il.first;

	while (ptr != NULL)
	{
		os << ptr -> data << " ";
		ptr = ptr -> link;
	}

	os << endl;

	return os;
}

void IntList::Push_Back(int e)
{
	if (!first)
		first = last = new Node(e);

	else
	{
		last -> link = new Node(e);
		last = last -> link;
	}
}

void IntList::Push_Front(int e)
{
	Node * newbie = new Node(e);
	newbie -> link = NULL;
	
	if (!first)
	{
		first = newbie;
		last = newbie;
	}
	
	else
	{
		newbie -> link = first;
		first = newbie;
	}
}



void IntList::Insert(int e)
{
	// Push_Front, Push_Back 사용할 것, 중복허용 안함
	if (!first)
	{
		// 빈 리스트인 경우
		Push_Front(e);
	}

	else if (first -> data > e)
	{
		// 리스트 맨 앞에 추가
		Push_Front(e);
	}

	else if (first -> data != e)
	{
		Node * a = first -> link;
		Node * b = first;
		
		while (a != NULL)
		{
			if (e > (a -> data))
			{
				b = a;
				a = a -> link;
			}
			
			else if (e < a -> data)
			{
				Node * newbie = new Node(e);
				newbie -> link = NULL;
			
				newbie -> link = b -> link;
				b -> link = newbie;
				
				break;
			}		
			
			else if (e == a -> data)
			{
				break;
			}
		}
		
		if (a == NULL)
		{
			Push_Back(e);	
		}
	}
}

void IntList::Delete(int e)
{
	if (first -> data == e)
	{
		// 첫 번째 데이터를 삭제할 경우
		Node * del_ptr = first;
		first = first -> link;
		delete del_ptr;
	}

	else
	{
		Node * a = first -> link;
		Node * b = first;
		
		while (a != NULL)
		{
			if (e == a -> data)
			{
				b -> link = a -> link;
				delete a;
				a = b;
				
				break;
			}
			
			b = a;
			a = a -> link;
		}
		
		if (a == NULL)
		{
			last = b;
		}
	}
}
\end{lstlisting}
\begin{itemizeReduced}
    \item[/*] 
    \item[*] line 51 - 101: 데이터 입력 매개변수 e 를 node 의 data 로써 받아와, 각 조건에 맞게 Insert 를 구현한 함수이다. 
    \item[*] first 인 경우, first 의 data 가 e 보다 큰 경우, first 의 data 가 e 와 다를 경우로 조건을 나누며, 후자의 경우 안에서는 계속해서 리스트의 node data 들과 e 의 대소관계를 따져나가며 data 가 오름차순으로 적절히 정렬되도록 구현하였다. 
    \item[*] 입력값 e 에 대하여 각 node 들 사이에서의 대소관계를 앞 뒤로 따지기 위해서 line 115 - 116 과 같이 first 와 first -> link 를 가리키는 node type pointer a 와 b 를 생성하였고, 이는 while 문을 통해 한칸씩 뒤로 이동한다.
    \item[*] 한편, Push\_Front, Push\_Back 함수를 각 조건에 맞는 부분에 사용하였으며, data 의 중복을 허용하지 않게 구현하였다.
    \item[*] line 103 - 138: list 의 node 를 입력 매개변수 e 에 대하여 각 조건에 맞게 delete 를 구현한 함수이다. 
    \item[*] line 108: 매개변수 e 를 data 로 가지는 node 를 삭제할 때, 해당 node 를 가리키는 pointer 를 생성한다.
    \item[*] line 109: first -> data == e 의 경우이므로, first = first -> link 라 한다.
    \item[*] line 110: 그제서야 삭제하고자 하는 node 를 del\_ptr 을 이용하여 지울 수 있다.
    \item[*] line 113 - 137: first -> data != e 의 경우 delete 구현하는 코드이며, 이는 동일 조건의 Insert 함수 구현과 마찬가지로, first 와 first -> link 를 가리키는 node type pointer a, b 생성 및 while 문 통해 a != NULL 일 때까지 한 칸씩 갱신하여 반복한다. 
    \item[*] 다만 delete 는 대소관계를 따질 필요 없이 입력받은 값이 해당 리스트 node data 중 일치하는지의 여부만 판단하면 되기 때문에, e == a -> data 의 경우만 조사한다.
\end{itemizeReduced}
*/

%------------------------------------------------

\newpage

\subsection{Results}
\subsubsection{makefile}
\begin{lstlisting} [basicstyle=\footnotesize]
hw6:	hw6.o list.o
	g++ -o hw6 hw6.o list.o
hw6.o list.o:	list.h
\end{lstlisting}

\subsubsection{input.in}
\begin{lstlisting} [basicstyle=\footnotesize]
1 5 4 4 3 2
-1
1 3 2 2 5
-1
1 2
3 4
\end{lstlisting}

\subsubsection{compile}
\begin{lstlisting} [basicstyle=\footnotesize]
[B711222@localhost hw6d]$ ./hw6 < input.in
============ Input =============
1
1 5
1 4 5
1 4 5
1 3 4 5
1 2 3 4 5
============= Delete ============
2 3 4 5
2 4 5
4 5
4 5
4
========== Push_Front ===========
1 4
2 1 4
=========== Push_Back ===========
2 1 4 3
2 1 4 3 4
\end{lstlisting}

\begin{itemizeReduced}
    \item[/*] 어려웠던 점
    \item[*] 결과값이 조금씩 흐트러져서 나왔었는데, Push\_Front 와 Delete 함수에서 last 에 대한 정의를 해주지 않았기 때문이라는 것을 캐치해야했던 점이 어려웠다.
\end{itemizeReduced}
*/

\newpage
\begin{figure}
\centering 
\includegraphics[scale=0.8]{linkedlist.png}
\includegraphics[scale=0.5]{linkedlist2.png}
\end{figure} 

%----------------------------------------------------------------------------------------
%	REFERENCE LIST
%----------------------------------------------------------------------------------------


%----------------------------------------------------------------------------------------

\end{document}
