%%%%%%%%
% Journal Article
% LaTeX Template
% Version 1.4 (15/5/16)
%
% This template has been downloaded from:
% http://www.LaTeXTemplates.com
%
% Original author:
% Frits Wenneker (http://www.howtotex.com) with extensive modifications by
% Vel (vel@LaTeXTemplates.com)
%
% License:
% CC BY-NC-SA 3.0 (http://creativecommons.org/licenses/by-nc-sa/3.0/)
%
%%%%%%%%%%%%%%%%%%%%%%%%%%%%%%%%%%%%%%%%%

%----------------------------------------------------------------------------------------
%	PACKAGES AND OTHER DOCUMENT CONFIGURATIONS
%----------------------------------------------------------------------------------------

\documentclass[twoside,twocolumn]{article}

\usepackage[utf8]{inputenc}
\usepackage[hangul]{kotex}

\newenvironment{itemizeReduced}{
\begin{list}{\labelitemi}{\leftmargin=1em}
\setlength{\itemsep}{1pt}
\setlength{\parskip}{0pt}
\setlength{\parsep}{0pt}}{\end{list}
}

\usepackage{listings}
\usepackage{xcolor}
\lstset{
      language = C++,
      numbers=left,
      stepnumber=1,
      basicstyle=\ttfamily,
      columns=fullflexible,
      breaklines=true,
      postbreak=\mbox{\textcolor{pink}{$\hookrightarrow$}\space}
      escapeinside={\%*}{*)},          % if you want to add LaTeX within your code
      frame=tb, % draw a frame at the top and bottom of the code block
      tabsize=4, % tab space width
      showstringspaces=false, % don't mark spaces in strings
      commentstyle=\color{yellow}, % comment color
      keywordstyle=\color{blue}, % keyword color
      stringstyle=\color{green},
}
\usepackage{blindtext} % Package to generate dummy text throughout this template 

\usepackage[sc]{mathpazo} % Use the Palatino font
\usepackage[T1]{fontenc} % Use 8-bit encoding that has 256 glyphs
\linespread{1.13} % Line spacing - Palatino needs more space between lines
\usepackage{microtype} % Slightly tweak font spacing for aesthetics

\usepackage[english]{babel} % Language hyphenation and typographical rules

\usepackage[hmarginratio=1:1,top=32mm,columnsep=20pt]{geometry} % Document margins
\usepackage[hang, small,labelfont=bf,up,textfont=it,up]{caption} % Custom captions under/above floats in tables or figures
\usepackage{booktabs} % Horizontal rules in tables

\usepackage{lettrine} % The lettrine is the first enlarged letter at the beginning of the text

\usepackage{enumitem} % Customized lists
\setlist[itemize]{noitemsep} % Make itemize lists more compact

\usepackage{abstract} % Allows abstract customization
\renewcommand{\abstractnamefont}{\normalfont\bfseries} % Set the "Abstract" text to bold
\renewcommand{\abstracttextfont}{\normalfont\small\itshape} % Set the abstract itself to small italic text

\usepackage{titlesec} % Allows customization of titles
\renewcommand\thesection{\Roman{section}} % Roman numerals for the sections
\renewcommand\thesubsection{\roman{subsection}} % roman numerals for subsections
\titleformat{\section}[block]{\large\scshape\centering}{\thesection.}{1em}{} % Change the look of the section titles
\titleformat{\subsection}[block]{\large}{\thesubsection.}{1em}{} % Change the look of the section titles

\usepackage{fancyhdr} % Headers and footers
\pagestyle{fancy} % All pages have headers and footers
\fancyhead{} % Blank out the default header
\fancyfoot{} % Blank out the default footer
% Custom header text
\fancyfoot[RO,LE]{\thepage} % Custom footer text

\usepackage{titling} % Customizing the title section

\usepackage{hyperref} % For hyperlinks in the PDF

%----------------------------------------------------------------------------------------
%	TITLE SECTION
%----------------------------------------------------------------------------------------

\setlength{\droptitle}{-4\baselineskip} % Move the title up

\pretitle{\begin{center}\Huge\bfseries} % Article title formatting
\posttitle{\end{center}} % Article title closing formatting
\title{자료구조 HW2} % Article title
\author{%
\normalsize B711222 \textsc{박조은} \\ % Your name
\normalsize Hongik University \\ % Your institution
\normalsize \href{mailto:mrnglory@mail.hongik.ac.kr}{mrnglory@mail.hongik.ac.kr} % Your email address
%\and % Uncomment if 2 authors are required, duplicate these 4 lines if more
%\textsc{Jane Smith}\thanks{Corresponding author} \\[1ex] % Second author's name
%\normalsize University of Utah \\ % Second author's institution
%\normalsize \href{mailto:jane@smith.com}{jane@smith.com} % Second author's email address
}
\date{\today} % Leave empty to omit a date
\renewcommand{\maketitlehookd}{%

}

%----------------------------------------------------------------------------------------

\begin{document}

% Print the title
\maketitle

%----------------------------------------------------------------------------------------
%	ARTICLE CONTENTS
%----------------------------------------------------------------------------------------

\section{List of Source Files}
\begin{itemizeReduced}
\item hw2a
\begin{itemizeReduced}
    \item polya.h
    \item polya.cpp
    \item hw2a.cpp
    \end{itemizeReduced}
\item hw2b
\begin{itemizeReduced}
    \item polyb.h
    \item polyb.cpp
    \item hw2b.cpp
    \end{itemizeReduced}
\end{itemizeReduced}
%------------------------------------------------

\section{hw2a}

\subsection{polya.h}
\begin{lstlisting} [basicstyle=\footnotesize]
#include <iostream>
#ifndef POLYNOMIAL_H
#define POLYNOMIAL_H
using namespace std;

class Polynomial;
class Term {
public:
        friend class Polynomial;
        friend ostream& operator << (ostream&, Polynomial&);
        friend istream& operator >> (istream&, Polynomial&);

private:
        float coef;
        int exp;
};

class Polynomial {
public:
        Polynomial();
        Polynomial operator + (Polynomial&);
        void NewTerm(const float, const int);
friend ostream& operator << (ostream&, Polynomial&);
friend istream& operator >> (istream&, Polynomial&);

private:
        Term *termArray;
        int capacity;
        int terms;
};
#endif
\end{lstlisting}
\begin{itemizeReduced}
    \item[/*] 
    \item[*] line 2, 3, 31: 헤더파일 중복으로 발생할 수 있는 문제를 막기 위해 \#ifndef \#endif \#define 전처리기 사용
    \item[*] line 7: Term class 생성
    \item[*] line 6: 밑의 class Term{} 에서 Polynomial 선언 가능케 하도록 미리 호출
    \item[*] line 9 - 11: 다른 클래스의 friend로 선언된 멤버들간에 상호 접근할 수 있게 friend 키워드 선언
    \item[*] line 10: output operator overloading
    \item[*] line 11: input operator overloading
    \item[*] line 13 - 15: private 타입의 멤버변수 coef, exp 는 각각 polynomial 의 계수와 지수를 지칭
    \item[*] line 20: 다항식 p(x) = 0 을 생성
    \item[*] line 21: plus operator overloading
    \item[*] line 22: 새로운 항의 추가 및 배열의 크기를 두 배로 확장하는 기능을 가진 public 타입의 멤버함수 선언 \\ (polya.cpp file 의 line 71 - 84 에서 구현)
    \item[*] line 26 - 29: Polynomial의 전용 데이터 멤버 선언
    \item[*] line 27: 0 이 아닌 항의 배열
    \item[*] line 28: termArray 의 크기
    \item[*] line 29: 0 이 아닌 항의 갯수
\end{itemizeReduced}
*/

\newpage

\subsection{polya.cpp}
\begin{lstlisting} [basicstyle=\footnotesize]
#include <iostream>
#include "polya.h"
using namespace std;

istream& operator >> (istream& is, Polynomial& p)
{
        int noofterms;
        float coef;
        int exp;

        is >> noofterms;

        for (int i = 0; i < noofterms; i++)
        {
                is >> coef >> exp;
                p.NewTerm(coef, exp);
        }
        return is;
}

ostream& operator << (ostream& os, Polynomial& p)
{
        for (int i = 0; i < p.terms; i++)
        {
                if (p.termArray[i].coef > 0)
                {
                        if (i != 0)
                        {
                                if (p.termArray[i].coef != 1)
                                        os << "+" << p.termArray[i].coef;
                                else
                                        os << "+";
                        }
                        else
                                if (p.termArray[i].coef != 1)
                                        os << p.termArray[i].coef;
                }

                if (p.termArray[i].coef < 0)
                {
                        if (p.termArray[i].coef != -1)
                                os << p.termArray[i].coef;
                        else
                                os << "-";
                }

                if (p.termArray[i].exp != 0)
                        os << "x^" << p.termArray[i].exp;

                for (int j = i + 1; j < p.terms; j++)
                {
                        if (p.termArray[i].exp == p.termArray[j].exp)
                        {
                                p.termArray[i].coef += p.termArray[j].coef;
                                p.termArray[j] = p.termArray[p.terms - 1];
                                p.terms--;
                        }
                }

        }

        os << endl;
        return os;
}

Polynomial::Polynomial():capacity(1), terms(0)
{
        termArray = new Term[capacity];
}

void Polynomial::NewTerm(const float theCoef, const int theExp) // theCoeff -> theCoef
{
        if (terms == capacity)
        {
                capacity *= 2;
                Term * temp = new Term [capacity];
                copy(termArray, termArray + terms, temp);
                delete [] termArray;
                termArray = temp;
        }

        termArray[terms].coef = theCoef;
        termArray[terms++].exp = theExp;
}

Polynomial Polynomial::operator + (Polynomial& b)
{
        Polynomial c;
        int aPos = 0, bPos = 0;

        while ((aPos < terms) && (bPos < b.terms))
        {
                if (termArray[aPos].exp == b.termArray[bPos].exp)
                {
                        float t = termArray[aPos].coef + b.termArray[bPos].coef;
                        if (t)
                                c.NewTerm(t, termArray[aPos].exp);
                        aPos++;
                        bPos++;
                }
                else if (termArray[aPos].exp < b.termArray[aPos].exp)
                {
                        c.NewTerm(b.termArray[bPos].coef, b.termArray[bPos].exp);
                        bPos++;
                }
                else
                {
                        c.NewTerm(termArray[aPos].coef, termArray[aPos].exp);
                        aPos++;
                }
        }

        for (; aPos < terms; aPos++)
                c.NewTerm(termArray[aPos].coef, termArray[aPos].exp);

        for (; bPos < terms; bPos++)
                c.NewTerm(b.termArray[bPos].coef, b.termArray[bPos].exp);

        for (int i = 0; i < c.terms; i++)       // 내림차순 정렬
                for (int j = i + 1; j < c.terms; j++)
                        if (c.termArray[i].exp < c.termArray[j].exp)
                        {
                                Term temp = c.termArray[i];
                                c.termArray[i] = c.termArray[j];
                                c.termArray[j] = temp;
                        }

        return c;
}
\end{lstlisting}
\begin{itemizeReduced}
    \item[/*] 
    \item[*] line 5 - 19: poly.in file 에 적힌 값을 불러 들여서 각각 항의 갯수, 계수 및 지수라는 정보로 저장하는 과정
    \item[*] line 7: 항의 갯수 정보 값 저장 공간
    \item[*] line 8: 각 항의 계수 정보 값 저장 공간
    \item[*] line 9: 각 항의 지수 정보 값 저장 공간
    \item[*] line 13: 각 항의 계수 및 지수 정보를 들여오는 반복문의 횟수를 해당 다항식의 항의 갯수인 noofterms 의 값 만큼 실행
    \item[*] line 21 - 64: 입력 받은 polynomial 을 실제 수식의 형태로 경우에 따라 적절히 출력 해주는 함수 구현
    \item[*] line 23: polynomial 의 항의 갯수 만큼 반복문 수행
    \item[*] line 25: 계수가 양수일 경우
    \item[*] line 27: 초항이 아닐 경우
    \item[*] line 29 - 30: 계수가 1 이 아닐 경우, + 기호와 계수를 함께 출력한다.
    \item[*] line 31: 계수가 1 일 경우, 계수를 출력하지 않고 + 기호만 출력한다.
    \item[*] line 34: 초항일 경우
    \item[*] line 35 - 36: 계수가 1이 아닐 경우, + 기호 없이 계수만 출력한다.
    \item[*] line 39: 계수가 음수일 경우
    \item[*] line 41: 계수가 -1 이 아닐 경우
    \item[*] line 42: 계수에 - 부호가 이미 붙어있으므로 계수 자체를 출력한다.
    \item[*] line 43: 계수가 -1 일 경우
    \item[*] line 44: - 부호만 붙인다.
    \item[*] line 47: 지수가 0 이 아닐 경우, 즉 상수항이 아닐 경우
    \item[*] line 48: x 의 exp 승의 표현 형태를 출력한다.
    \item[*] line 50: 같은 차수의 항끼리 합쳐서 하나의 항으로 표현하는 코드
    \item[*] line 52: 서로 비교한 두 개의 항이 같은 차수일 경우
    \item[*] line 54: 두 항의 계수를 합하여 index i 에 저장, 고로 하나의 항으로 합치기 위한 과정
    \item[*] line 55: 차수가 같은 두 항 중 index j 항은 이미 index i 항과 합쳤으므로, 더 이상 고려 대상이 아니다. 따라서 index j 항을 삭제 해야 하며, 다항식 맨 뒤에 존재하고 있는 항, 즉 index p.terms - 1 항을 index j 항에 복사하여 index j 항을 날린다.
    \item[*] line 56: 그리고 이미 index j 항에 복사 한 index p.terms - 1 항을 제거한다.
    \item[*] line 66: Polynomial 의 기정 생성자가 capacity 와 terms 를 각각 0 과 1 로 초기화한다.
    \item[*] line 68: capacity 의 크기인 Term 타입의 배열 termArray
    \item[*] line 71: 새로운 항을 termArray 끝에 추가한다.
    \item[*] line 73: termArray 가 꽉 찰 경우
    \item[*] line 75: 배열의 크기를 나타내는 값 capacity 를 두 배로 확장한다.
    \item[*] line 76: 크기가 capacity 인 Term 타입의 배열 temp 를 임시로 생성한다.
    \item[*] line 77: 배열 temp 에 termArray 의 모든 원소들을 복사한다.
    \item[*] line 78: termArray 의 이전 메모리를 반환한다.
    \item[*] line 79: 배열 temp 에 임시로 복사해둔 모든 항들을 다시 배열 termArray 에 저장한다.
    \item[*] line 82: 배열 termArray 에 새로운 계수 정보를 추가한다.
    \item[*] line 83: 배열 termArray 의 항의 갯수를 증가시킨 자리에 새로운 지수 정보를 추가한다.
    \item[*] line 86: *this 와 b 를 더한 결과를 반환한다.
    \item[*] line 88: Polynomial 타입의 다항식 c 를 생성한다.
    \item[*] line 89: 다항식 *this 와 b 의 각 항들의 계수 및 지수를 비교하기 위해 index 값을 나타내는 a position, b position 변수를 설정 및 초기화한다.
    \item[*] line 93: *this 와 b 가 같은 차수의 항이 있을 경우
    \item[*] line 95: 각 계수의 값들을 더한 결과를 실수형 변수 t 에 저장한다.
    \item[*] line 96: 계수가 t 인 해당 차수 항을 다항식 c 에 새로운 항으로 추가한다.
    \item[*] line 98, 99: aPos, bPos 를 1씩 증가시킨다.
    \item[*] line 101 - 109: 다항식 c 에 차수를 내림차순으로 차곡차곡 저장한다.
    \item[*] line 113 - 114: *this의 나머지 항들을 다항식 c 에 추가한다.
    \item[*] line 116 - 117: 다항식 b 의 나머지 항들을 다항식 c 에 추가한다.
    \item[*] line 119 - 126: c 에 새로 추가된 항들을 다시 전체적으로 내림차순 정렬 시켜준다.
\end{itemizeReduced}
*/

\newpage

\subsection{hw2a.cpp}
\begin{lstlisting} [basicstyle=\footnotesize]
#include <iostream>
using namespace std;
#include "polya.h"
int main()
{
        Polynomial p1, p2;

        cin >> p1 >> p2;
        Polynomial p3 = p1 + p2;
        cout << p1 << p2 << p3;
}
\end{lstlisting}
\begin{itemizeReduced}
    \item[/*]
    \item[*] line 6: Polynomial class 의 instance 로 p1, p2 를 정의한다.
    \item[*] line 8: instance p1, p2 에 해당되는 다항식을 구성하는 값들을 입력 받아온다.
    \item[*] line 9: 입력받은 두 개의 다항식을 덧셈하여 세로운 다항식에 저장한다.
    \item[*] line 10: instance p1, p2 에 해당되는 다항식 및 다항식 덧셈 결과를 출력한다.
\end{itemizeReduced}
*/
%------------------------------------------------

\subsection{Results}
\subsubsection{makefile}
\begin{lstlisting} [basicstyle=\footnotesize]
hw2a:hw2a.o polya.o
        g++ -o hw2a hw2a.o polya.o
hw2a.o polya.o:polya.h
\end{lstlisting}

\subsubsection{poly.in}
\begin{lstlisting} [basicstyle=\footnotesize]
3       3.0 5   2.0 3   -4.0 0
4       1.0 8   -7.0 5  -1.0 3  -3.0 0
\end{lstlisting}

\subsubsection{compile}
\begin{lstlisting} [basicstyle=\footnotesize]
[B711222@localhost hw2d]$ ./hw2a < poly.in
3x^5+2x^3-4
x^8-7x^5-x^3-3
x^8-4x^5+x^3-7
\end{lstlisting}

\clearpage

\section{hw2b}
\subsection{polyb.h}
\begin{lstlisting} [basicstyle=\footnotesize]
#ifndef POLYNOMIAL_H
#define POLYNOMIAL_H
#include <iostream>
using namespace std;

class Polynomial;
class Term {
friend class Polynomial;
friend ostream& operator << (ostream&, Polynomial&);
friend istream& operator >> (istream&, Polynomial&);

private:
        float coef;
        int exp;
};

class Polynomial {
public:
        Polynomial();
        Polynomial operator + (Polynomial&);
        Polynomial operator * (Polynomial&);
        void NewTerm(const float, const int);

friend ostream& operator << (ostream&, Polynomial&);
friend istream& operator >> (istream&, Polynomial&);

private:
        Term *termArray;
        int capacity;
        int terms;
};
#endif
\end{lstlisting}
\begin{itemizeReduced}
    \item[/*] hw2a 와 동일한 부분은 설명을 생략하도록 한다.
    \item[*] line 21: multiply operator overloading
\end{itemizeReduced}
*/

\subsection{polyb.cpp}
\begin{lstlisting} [basicstyle=\footnotesize]
#include <iostream>
#include "polyb.h"
using namespace std;

istream& operator >> (istream& is, Polynomial& p)
{
        int noofterms;
        float coef;
        int exp;

        is >> noofterms;

        for (int i = 0; i < noofterms; i++)
        {
                is >> coef >> exp;
                p.NewTerm(coef, exp);
        }
        return is;
}

ostream& operator << (ostream& os, Polynomial& p)
{
        for (int i = 0; i < p.terms; i++)
        {
                if (p.termArray[i].coef > 0)
                {
                        if (i != 0)
                        {
                                if (p.termArray[i].coef != 1)
                                        os << "+" << p.termArray[i].coef;
                                else
                                        os << "+";
                        }
                        else
                                if (p.termArray[i].coef != 1)
                                        os << p.termArray[i].coef;
                }

                if (p.termArray[i].coef < 0)
                {
                        if (p.termArray[i].coef != -1)
                                os << p.termArray[i].coef;
                        else
                                os << "-";
                }

                if (p.termArray[i].exp != 0)
                        os << "x^" << p.termArray[i].exp;
        }

        for (int i = 0; i < p.terms; i++)
                for (int j = i + 1; j < p.terms; j++)
                {
                        if (p.termArray[i].exp < p.termArray[j].exp)
                        {
                                Term temp = p.termArray[i];
                                p.termArray[i] = p.termArray[j];
                                p.termArray[j] = temp;
                        }
                }

        os << endl;
        return os;
}

Polynomial::Polynomial():capacity(1), terms(0)
{
        termArray = new Term[capacity];
}

void Polynomial::NewTerm(const float theCoef, const int theExp) // theCoeff -> theCoef
{
        if (terms == capacity)
        {
                capacity *= 2;
                Term * temp = new Term [capacity];
                copy(termArray, termArray + terms, temp);
                delete [] termArray;
                termArray = temp;
        }

        termArray[terms].coef = theCoef;
        termArray[terms++].exp = theExp;
}

Polynomial Polynomial::operator + (Polynomial& b)
{
        Polynomial c;
        int aPos = 0, bPos = 0;

        while ((aPos < terms) && (bPos < b.terms))
        {
                if (termArray[aPos].exp == b.termArray[bPos].exp)
                {
                        float t = termArray[aPos].coef + b.termArray[bPos].coef;
                        if (t)
                                c.NewTerm(t, termArray[aPos].exp);
                        aPos++;
                        bPos++;
                }
                else if (termArray[aPos].exp < b.termArray[aPos].exp)
                {
                        c.NewTerm(b.termArray[bPos].coef, b.termArray[bPos].exp);
                        bPos++;
                }
                else
                {
                        c.NewTerm(termArray[aPos].coef, termArray[aPos].exp);
                        aPos++;
                }
        }

        for (; aPos < terms; aPos++)
                c.NewTerm(termArray[aPos].coef, termArray[aPos].exp);

        for (; bPos < terms; bPos++)
                c.NewTerm(b.termArray[bPos].coef, b.termArray[bPos].exp);

        return c;
}

Polynomial Polynomial::operator * (Polynomial& b)
{
        Polynomial c;
        int aPos = 0, bPos = 0;

        for (aPos = 0; aPos < terms; aPos++)
        {
                for (bPos = 0; bPos < b.terms; bPos++)
                {
                        int tempCoef = termArray[aPos].coef * b.termArray[bPos].coef;
                        int tempExp = termArray[aPos].exp + b.termArray[bPos].exp;

                        c.NewTerm(tempCoef, tempExp);
                }
        }

        for (int i = 0; i < c.terms; i++)
        {
                for (int j = i + 1; j < c.terms; j++)
                {
                        if (c.termArray[i].exp == c.termArray[j].exp)
                        {
                                c.termArray[i].coef += c.termArray[j].coef;
                                c.termArray[j] = c.termArray[c.terms - 1];
                                c.terms--;
                        }
                }
        }

        for (int i = 0; i < c.terms; i++)       // 내림차순 정렬
                for (int j = i + 1; j < c.terms; j++)
                        if (c.termArray[i].exp < c.termArray[j].exp)
                        {
                                Term temp = c.termArray[i];
                                c.termArray[i] = c.termArray[j];
                                c.termArray[j] = temp;
                        }

        return c;
}
\end{lstlisting}
\begin{itemizeReduced}
    \item[/*] hw2a 와 동일한 부분은 설명을 생략하도록 한다.
    \item[*] line 122: *this 와 b 를 곱한 결과를 반환한다.
    \item[*] line 138 - 149: 곱셈을 통해 같은 차수의 항이 발생하는 경우를 처리하기 위하여, 각 항들을 하나의 항으로 합친 뒤 연산 후 더 이상 필요하지 않은 항들을 제거하는 코드를 다항식 곱셈 함수에 다시 작성하였다.
\end{itemizeReduced}
*/

\subsection{hw2b.cpp}
\begin{lstlisting} [basicstyle=\footnotesize]
#include <iostream>
using namespace std;
#include "polyb.h"

int main()
{
        Polynomial p1, p2;

        cin >> p1 >> p2;
        Polynomial p3 = p1 * p2;
        cout << p1 << p2 << p3;
}
\end{lstlisting}
\begin{itemizeReduced}
    \item[/*] hw2a 와 동일한 부분은 설명을 생략하도록 한다.
    \item[*] line 10 - 11: 다항식 곱셈 연산 결과를 Polynomial 타입의 p3 에 새로이 저장 및 각 다항식들을 출력한다.
\end{itemizeReduced}
*/

\subsection{results}
\subsubsection{makefile}
\begin{lstlisting} [basicstyle=\footnotesize]
hw2b:hw2b.o polyb.o
        g++ -o hw2b hw2b.o polyb.o
hw2b.o polyb.o:polyb.h
\end{lstlisting}

\subsubsection{poly.in}
\begin{lstlisting} [basicstyle=\footnotesize]
3       3.0 5   2.0 3   -4.0 0
4       1.0 8   -7.0 5  -1.0 3  -3.0 0
\end{lstlisting}

\subsubsection{compile}
\begin{lstlisting} [basicstyle=\footnotesize]
[B711222@localhost hw2d]$ ./hw2b < poly.in
3x^5+2x^3-4
x^8-7x^5-x^3-3
3x^13+2x^11-21x^10-21x^8-2x^6+19x^5-2x^3+12
\end{lstlisting}
%----------------------------------------------------------------------------------------
%	REFERENCE LIST
%----------------------------------------------------------------------------------------


%----------------------------------------------------------------------------------------

\end{document}
